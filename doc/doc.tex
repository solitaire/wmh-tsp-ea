\documentclass[12pt, a4paper]{article}
\usepackage[utf8]{inputenc}
\usepackage{polski}
\usepackage{hyperref}
\usepackage{float}
\usepackage{algorithm}
\usepackage{algpseudocode}
\usepackage{geometry}
\title{\textbf{Zastosowanie algorytmu ewolucyjnego do rozwiązywania problemu komiwojażera}}
\author{Anna Stępień \\ Adam Stelmaszczyk}
\date{}
\setlength{\parindent}{0in}

\begin{document}
\maketitle

\section{Zadanie}
Celem zadania jest zaprojektowanie algorytmu ewolucyjnego, który zostanie wykorzystany do rozwiązania problemu komiwojażera. \\

W ramach zadania zostanie zaprojektowana reprezentacja rozwiązania, metoda sukcesji, operatory genetyczne (mutacja, krzyżowanie).

Z punktu widzenia przydatności projektowanego algorytmu istotne jest przetestowanie go dla różnych parametrów, w szczególności dla:
\begin{itemize}
	\item różnych wartości prawdopodobieństwa mutacji i krzyżowania,
	\item grafów o różnej wielkości.
\end{itemize}

\section{Założenia}
Realizowana aplikacja będzie pracowała w trybie konsolowym i będzie przyjmowała
pliki z danymi przekazany na strumień wejściowy.

\subsection{Reprezentacja rozwiązań}

Wierzchołki grafu numerujemy od 1 do $n$. $n > 2$. Rozwiązanie reprezentujemy jako ciąg długości $n$, złożony z unikalnych liczb. 
Dla skrócenia zapisu stosujemy zapis 123, zamiast $(1,2,3)$. Przykładowo, w grafie pełnym o 3 wierzchołkach możliwych jest $3!$ rozwiązań: 123, 132, 213, 231, 312, 321.
Warto zauważyć, że połowę tych rozwiązań można pominąć, ze względu na to, że koszt przejścia z $a$ do $b$ jest zawsze 
taki sam jak z $b$ do $a$. Można zatem pominąć rozwiązania $231, 312, 321$. Pozostałe trzy rozwiązania mają identyczny koszt,
zatem dwa z nich również można pominąć. Tym samym jedyne interesujące nas rozwiązanie to $123$. W ogólnym przypadku, dla grafu pełnego
o $n$ wierzchołkach, liczba interesujących nas rozwiązań wynosi $\frac{(n-1)!}{2}$. Spośród redundantnych rozwiązań pozostawiane będą te,
których reprezentacja liczbowa jest najmniejsza. Przykładowo, dla grafu pełnego o $4$ wierzchołkach zamiast rozpatrywać wszystkie
24 rozwiązania, będziemy rozpatrywać tylko 3: 1234, 1324, 1342.

\subsection{Funkcja celu}

Wartość funkcji celu dla rozwiązania to suma wag krawędzi w cyklu dla danego rozwiązania.

\subsection{Operator mutacji}

Operator mutacji otrzymuje na wejściu rozwiązanie długości $n$ i zwraca również rozwiązanie długości $n$.

\begin{enumerate}
 \item Wylosuj indeks $i$ od 0 do $n-1$ zgodnie z rozkładem jednostajnym.
 \item Wylosuj indeks $j$ od 0 do $n-1$, ale różny od $i$, zgodnie z rozkładem jednostajnym.
 \item Zamień liczbę na pozycji $i$-tej w wejściowym ciągu z liczbą na pozycji $j$-tej.
\end{enumerate}

\subsection{Operator krzyżowania}

Operator krzyżowania na wejściu przyjmuje dwa rozwiązania długości $n$ (rodziców) i zwraca również dwa rozwiązania długości $n$ (potomków). 

Jako operator krzyżowania zostanie wykorzystany operator krzyżowania {PMX} (z częściowym odwzorowaniem).

Operator krzyżowania jest opisany dwoma punktami krzyżowania: $i$, $j$
\begin{enumerate}
	\item Dla sekwencji od $i$ o $i+j$ dokonaj wymiany,
	\item Skopiuj pozostałe części sekwencji (od $0$ do $i-1$ i od $j+1$ do $n$), dokonując niezbędnych zmian, aby rozwiązania potomne były permutacjami.
\end{enumerate}

\subsection{Metoda sukcesji}

Metoda sukcesji na wejściu otrzymuje dwie populacje o rozmiarze $NP$: starą oraz nową. Wyjściem jest jedna populacja o rozmiarze $NP$.

% może lepiej zamienić to na pseudokod %
\begin{enumerate}
 \item Dla $i$ od 0 do $NP-1$:
 \item Na pozycji $i$ w wyjściowej populacji ustaw rozwiązanie lepsze z pary stara[i], nowa[i].
\end{enumerate}

\subsection{Schemat algorytmu ewolucyjnego}

% pseudokod jak u Arabasa co jest po czym po kolei %

\nocite{*}
\bibliographystyle{plain}
\bibliography{references}
\end{document}
