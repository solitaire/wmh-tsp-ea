\documentclass[12pt, a4paper]{article}
\usepackage[utf8]{inputenc}
\usepackage{polski}
\usepackage{hyperref}
\usepackage{float}
\usepackage{geometry}
\title{\textbf{Zastosowanie algorytmu ewolucyjnego do rozwiązywania problemu komiwojażera}}
\author{Anna Stępień \\ Adam Stelmaszczyk}
\date{}
\setlength{\parindent}{0in}

\newgeometry{tmargin=2.5cm, bmargin=2.5cm, lmargin=3cm, rmargin=3cm}

\begin{document}
\maketitle

\section{Zadanie}
Celem zadania jest zaprojektowanie algorytmu ewolucyjnego, który zostanie wykorzystany do rozwiązania problemu komiwojażera. \\

W ramach zadania zostanie zaprojektowana reprezentacja rozwiązania, metoda sukcesji, operatory genetyczne (mutacja, krzyżowanie).

Z punktu widzenia przydatności projektowanego algorytmu istotne jest przetestowanie go dla różnych parametrów, w szczególności dla:
\begin{itemize}
	\item różnych wartości prawdopodobieństwa mutacji i krzyżowania,
	\item grafów o różnej wielkości.
\end{itemize}

\section{Założenia}
Realizowana aplikacja będzie pracowała w trybie konsolowym i będzie przyjmowała
pliki z danymi przekazany na strumień wejściowy.

\nocite{*}
\bibliographystyle{plain}
\bibliography{references}
\end{document}
